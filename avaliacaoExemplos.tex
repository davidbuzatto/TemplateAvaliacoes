% !TeX document-id = {52ccfad0-b367-4734-939e-4350e7d34ffb}
% !TeX TXS-program:compile = txs:///pdflatex/[--shell-escape]

\documentclass[
    12pt,     
    openright,
    twoside,  
    a4paper,  
    english,  
    brazil,   
    %draft
]{memoir}

\usepackage[c]{estrutura}

%\temaClaro
\temaEscuro

\curso{Bacharelado em Ciência da Computação}
\nomeDisciplina{Construção de Algoritmos e Programação}
\siglaDisciplina{CALC1}
\professorDisciplina{Prof. Dr. David Buzatto}

\data{25/04/2022}
\hora{13h00 às 15h30}


\begin{document}

\mainmatter

\FloatBarrier
\begin{table}[!htbp]
    \centering%
    \noindent%
    \large%
    \begin{tabularx}{\textwidth}{ c C{15cm} }
        \multirow{7}{*}{\includegraphics[scale=0.5]{imagens/logotipoIFSP}} & \\
        & {
            \Large%
            \scshape%
            \imprimircurso
        }\\%
        & \\%
        & Avaliação de \imprimirNomeDisciplina{ }(\imprimirSiglaDisciplina)\\%
        & \imprimirProfessorDisciplina\\%
        & \imprimirData{ }--{ }\imprimirHora\\%
        & \\%
    \end{tabularx}
\end{table}
\FloatBarrier

\begin{tabular}{ c p{8cm} c p{3cm} c p{1.5cm} }
    Nome: & \hrulefill & Prontuário: & \hrulefill & Nota: & \hrulefill \\
\end{tabular}

\textbf{Atenção:} para entregar as questões da avaliação, crie um pacote no JJudge e faça o envio na atividade da avaliação disponibilizada no Moodle.



\begin{questaoProgramacao}{1,0}{Detalhe da entrada esperada}{Detalhe da saída esperada}
    Esse é um exemplo de questão de programação
\end{questaoProgramacao}



\begin{questaoDissertativa}{1,0}
    Esse é um exemplo de questão dissertativa.
    
    \gerarLinhas{5}   
\end{questaoDissertativa}



\begin{questaoProgramacao}{1,0}{Primeiro numero: 7.5\\Segundo numero: 3.5}{7.50 + 3.50 = 11.00\\7.50 - 3.50 = 4.00\\7.50 * 3.50 = 26.25\\7.50 / 3.50 = 2.14}
    Escreva um programa que peça para o usuário fornecer o valor de dois números decimais. O programa deve usar o valor dos números para calcular o valor das quatro operações aritméticas básicas (adição, subtração, multiplicação e divisão). O resultado de cada operação deve ser armazenado em uma variável diferente. No final, o programa deve exibir ao usuário o resultado de cada operação, formatados usando duas casas decimais.
\end{questaoProgramacao}



\begin{questaoProgramacao}{1,0}{Altura: 5}{*\\
**\\
***\\
****\\
*****}
    Escreva um programa que leia um número inteiro no intervalo de 1, inclusive, até 20, inclusive. Esse valor representa a altura de um triângulo, que deverá ser desenhado de acordo com o padrão apresentado abaixo. Você deve usar, obrigatoriamente, a estrutura de repetição \texttt{for}. Se a altura fornecida estiver fora do intervalo, o programa deve avisar o usuário e pedi-la novamente.
\end{questaoProgramacao}
\entradaSaidaQuestaoE{Altura: -5\\Altura invalida, forneca-a novamente!\\Altura: 3}{*\\
**\\
***}



\begin{questaoProgramacao}{2,0}{}{}
    Utilize o modelo abaixo para testar a implementação das seguintes funções:
    
    \begin{enumerate}
    
        \item \inlineC{void somatorioMedia( const int a[], int n, int *somatorio, int *media )} (1,0):
        \begin{itemize}
            \item Calcula o somatório e a média aritmética dos valores de um array de inteiros.
            \begin{itemize}
                \item \destaque{a:} um array de inteiros;
                \item \destaque{n:} a quantidade de elementos do array;
                \item \destaque{somatorio:} um ponteiro para a variável que armazenará o somatório de todos os valores contidos no array após a execução;
                \item \destaque{media:} um ponteiro para a variável que armazenará a média de todos os valores contidos no array após a execução.
            \end{itemize}
        \end{itemize}
        
        \item \inlineC{int contarVogais( const char *str )} (1,0):
        \begin{itemize}
            \item Conta a quantidade de vogais, tanto minúsculas quanto maiúsculas, contidas em uma string e retorna a contagem.
            \begin{itemize}
                \item \destaque{str:} a string a ser verificada.
            \end{itemize}
        \end{itemize}
        
    \end{enumerate}
    
\begin{blocoC}[\normalsize]
#include <stdio.h>
#include <stdlib.h>
#include <stdbool.h>
#include <string.h>
#include <math.h>
#include <ctype.h>

void somatorioMedia( const int a[], int n, int *somatorio, int *media );
int contarVogais( const char *str );

int main() {

    int numSM[5] = { 4, 5, 3, 1, 7 };
    char strv[] = "Avaliacao de CALC1";
    int s;
    int m;
    int cv;

    somatorioMedia( numSM, 5, &s, &m );
    cv = contarVogais( strv );

    printf( "Somatorio: %d\nMedia: %d\n", s, m );
    printf( "Quantidade de vogais na string: %d\n", cv );

    return 0;

}

void somatorioMedia( const int a[], int n, int *somatorio, int *media ) {
    // sua implementação aqui...
}

int contarVogais( const char *str ) {
    // sua implementação aqui...
}
\end{blocoC}

\entradaSaidaQuestao{}{Somatorio: 20\\
Media: 4\\
Quantidade de vogais na string: 8}
    
\end{questaoProgramacao}



\begin{questaoProgramacao}{4,0}{}{}
    Utilize o modelo abaixo para testar a implementação das seguintes funções:
    
    \begin{enumerate}
    
        \item \inlineC{void converterParaMaiusculas( char *str )} (1,5):
        \begin{itemize}
            \item Gera a versão em maiúsculas de uma string qualquer.
            \begin{itemize}
                \item \destaque{str:} string que será processada.
            \end{itemize}
        \end{itemize}
        
        \item \inlineC{int compararCor( Cor *c1, Cor *c2 )} (2,5):
        \begin{itemize}
            \item Compara duas cores, retornando um valor negativo caso \destaque{c1} seja menor que \destaque{c2}, positivo caso \destaque{c1} seja maior que \destaque{c2} ou zero caso as duas cores sejam iguais.
            \begin{itemize}
                \item \destaque{c1:} um ponteiro para um cor;
                \item \destaque{c2:} um ponteiro para outra cor.
            \end{itemize}
            Considere que a estrutura \texttt{Cor} foi definida como:
\begin{blocoC}
typedef struct {
    int vermelho;
    int verde;
    int azul;
} Cor;
\end{blocoC}
            O primeiro critério de comparação de duas cores que deve ser adotado é o de que o componente vermelho tem prioridade, ou seja, quanto mais vermelho, maior a cor. O segundo critério é para o componente verde, quanto mais verde, maior. Por fim, deve ser adotado o mesmo critério para o componente azul. De forma análoga a uma data, o vermelho da cor corresponde ao ano de uma data, o verde ao mês e o azul ao dia.
                \end{itemize}
    
    \end{enumerate}

\begin{blocoC}[\normalsize]
#include <stdio.h>
#include <stdlib.h>
#include <stdbool.h>
#include <string.h>
#include <math.h>
#include <ctype.h>

typedef struct {
    int vermelho;
    int verde;
    int azul;
} Cor;

void converterParaMaiusculas( char *str );
int compararCor( Cor *c1, Cor *c2 );

int main() {

    char strMai[] = "Avaliacao de CALC1";
    Cor c1 = { .vermelho = 200, .verde = 100, .azul = 150 };
    Cor c2 = { .vermelho = 200, .verde = 150, .azul = 100 };

    converterParaMaiusculas( strMai );

    printf( "Versao em maiusculas: %s\n", strMai );

    printf( "Cores em ordem crescente:\n" );
    if ( compararCor( &c1, &c2 ) < 0 ) {
        printf( "rgb{%d, %d, %d} <= rgb{%d, %d, %d}\n",
               c1.vermelho, c1.verde, c1.azul,
               c2.vermelho, c2.verde, c2.azul );
    } else {
        printf( "rgb{%d, %d, %d} <= rgb{%d, %d, %d}\n",
               c2.vermelho, c2.verde, c2.azul,
               c1.vermelho, c1.verde, c1.azul );
    }

    return 0;

}

void converterParaMaiusculas( char *str ) {
    // sua implementação aqui...
}

int compararCor( Cor *c1, Cor *c2 ) {
    // sua implementação aqui...
}
\end{blocoC}

\entradaSaidaQuestao{}{Versao em maiusculas: AVALIACAO DE CALC1\\
Cores em ordem crescente:\\
rgb\{200, 100, 150\} <= rgb\{200, 150, 100\}}
    
\end{questaoProgramacao}



\arquivoC{Solução do Exercício Anterior}{fontes/solucaoExCor.c}



\begin{blocoC}[\Large]
int a;
int b;
\end{blocoC}

\begin{blocoC}
int a;
int b;
\end{blocoC}



{\raggedleft \textbf{Boa sorte!}\par}
\end{document}